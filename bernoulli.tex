\section{Positive Zeta Values}
    Part of the motivation behind the Bernoulli Numbers involves sums of the form 
    \[\sum_{i=1}^ni^k\] for $k\in \mathbb{Z}^{+}$. 
    These numbers show up frequently in analysis, and are useful for deriving expressions for 
    the Riemann Zeta function, $\zeta(s)=\sum_{n=1}^{\infty}\frac{1}{n^s}$. 
\begin{definition}
    $B_0=1,B_1=-\frac{1}{2}, B_2=\frac{1}{6}, B_3 = 0$. 
    Our first recursive definition of these numbers is:
    \[(m+1)B_m=-\sum_{i=0}^{m-1}{{m+1}\choose i} B_i\]
\end{definition}
It can be shown that this formulation has an interchangeable definition.
\begin{theorem}
    The Bernoulli Numbers can be modelled via a generating function. 
    \[\frac{t}{e^t-1}=\sum_{m=0}^{\infty}B_m\frac{t^m}{m!}\]
\end{theorem}

\begin{corollary}
    Oddly indexed Bernoulli Numbers are 0.
\end{corollary}
We now consider a Fourier Series approach to computing the Zeta function with Bernoulli Numbers. Let's first build up some
theory on Fourier Analysis.
Let $f:[a,b]\rightarrow \mathbb{C}$ be Riemann integrable. Set $L=b-a$, where $f$ is L-periodic.
\begin{definition}
    $\forall n\in\mathbb{Z}$ let the n-th Fourier coefficient be 
    \[
        \hat{f}(n)=\frac{1}{L}\int_a^bf(x)e^{-2\pi inx/L}\; dx
    \]
    and the Fourier series of $f$:
    \[
        f(x)=\sum_{n\in\mathbb{Z}}\hat{f}(n)e^{2\pi inx/L}  
    \]
\end{definition}
\begin{example}
    Compute $\hat{f}(n)$ for $\cos(\alpha x), x\in [-\pi,\pi], \alpha\in\mathbb{C}\backslash\mathbb{Z}$
\end{example}
\textbf{Solution:} We know $f(x)=\sum_{n\in\mathbb{Z}}\hat{f}(n)e^{i\pi n}$,
where $\hat{f}(n)=\frac{1}{2\pi}\int_{-\pi}^{\pi}f(x)e^{-inx}\;dx$.
So 
\[
    \hat{f}(n)=\frac{1}{2\pi}\int_{-\pi}^{\pi}\cos(\alpha x)e^{-inx}\;dx
    =\frac{1}{2\pi}\int_{-\pi}^{\pi}\frac{e^{i\alpha x}+e^{-i\alpha x}}{2}e^{-inx}\; dx
    \]
    \[= \frac{1}{4\pi}\big[\int_{-\pi}^{\pi}(e^{i\alpha x}+e^{-i\alpha x})e^{-inx}\; dx\big]
\]
Solving both integrals yields 
\[
    \hat{f}(n) = \frac{1}{4\pi}\big[\frac{e^{i\pi(\alpha-n)}-e^{-i\pi(\alpha-n)}}{i(\alpha-n)}+
    \frac{e^{-i\pi(\alpha+n)}-e^{\pi i(\alpha+n)}}{-i(\alpha+n)}\big]
\]  
\[
    =\frac{1}{4\pi}\big[\frac{2\sin(\pi(\alpha-n))}{\alpha-n}+\frac{2\sin(\pi(\alpha+n))}{\alpha+n}\big]
\]
Expansion into sums and differences of $\sin$ and $\cos$ leads to much cancellation and finally,
\[
    \hat{f}(n)=\frac{\alpha \sin(\pi \alpha)(-1)^n}{\pi(\alpha^2-n^2)}
\]  
\begin{theorem}
    Let $f:\mathbb{R}\rightarrow \mathbb{C}$ be riemann integrable on any closed interval, and 
    periodic with period $L$. Then the Fourier coefficients of $f$ don't depend on which $L$-interval
    is used to compute them.
\end{theorem}
\begin{proof}
    The periodicity conditions here trivialize the solution. Formally, integrating by parts will
    yield that the result reduces to showing 
    \[
        f(t)-f(t+L)=f(t_0)-f(t_0+L)
    \]
    which must be true, since they are both $0$ by periodicity.

\end{proof}

\begin{theorem}
    When $f$ is twice continuously differentiable on the circle $\mathbb{R}/2\pi\mathbb{Z}$, 
    it satisfies \[\sum_{n\in\mathbb{Z}}|\hat{f}(n)|<\infty\]
\end{theorem}

\begin{proof}
    We'll prove this by finding a form for the fourier coefficients, and bounding the resultant 
    by some constant. We have 
    \[
        \hat{f}(n)=\frac{1}{2\pi}\int_0^{2\pi}f(\theta)e^{-in\theta}\;d\theta 
        = \frac{1}{2\pi}f(\theta)\frac{-e^{-in\theta}}{in}\bigg |_0^{2\pi}+\frac{1}{2\pi i n}\int_0^{2\pi}
        f'(\theta)e^{-in\theta}\; d\theta 
    \]
    \[
        =\frac{1}{2\pi i n}\int_0^{2\pi}f'(\theta)e^{-in\theta}\;d\theta
    \]
    Integrating by parts again, this is equal to 
    \[
        \frac{1}{2\pi in}f'(\theta)\frac{-e^{-in\theta}}{in}\bigg |_0^{2\pi}+\frac{1}{2\pi(in)^2}
        \int_0^{2\pi}f''(\theta)e^{-in\theta}\; d\theta 
    \]
    \[ -\frac{1}{n^2}\int_0^{2\pi}f''(\theta)e^{-in\theta}\; d\theta
        \leq -\frac{1}{n^2}\int_0^{2\pi}|f''(\theta)|\;d\theta
    \]
    So \[
        |\hat{f}(n)|\leq \frac{1}{n^2}\int_0^{2\pi}|f''(\theta)\;d\theta
    \]
    Bound the integral by $B$, which must be constant by our assumptions of continuity and differentiability.
    Then 
    \[
        \sum_{n\in\mathbb{Z}}|\hat{f}(n)|\leq \sum_{n\in\mathbb{Z}}\frac{1}{n^2}\int_0^{2\pi}|f''(\theta)|\; 
        d\theta\leq \sum_{n\in\mathbb{Z}}(\frac{1}{n^2}\cdot B)=B\sum_{n\in\mathbb{Z}}\frac{1}{n^2}<\infty
    \]
\end{proof}
Now, lets consider these Bernoulli Numbers through the lense of Fourier Analysis. 
We begin with Bernoulli polynomials.
\begin{definition}
    Define the Bernoulli polynomials $B_n(x)$, $n\in \mathbb{Z}_{\geq 0}$ by the generating function
    \[\frac{te^{xt}}{e^{t}-1}=\sum_{k=0}^{\infty}B_k(x)\frac{t^k}{k!}\]
\end{definition}
We'll begin with a simple lemma about power series.
\begin{lemma}
    Let $f(t)=\sum_{n=0}^{\infty}a_nt^n$ and $g(t)=\sum_{n=0}^{\infty}b_nt^n$
    Then \[f(t)g(t)=\sum_{n=0}^{\infty}\big[(\sum_{k=0}^na_{n-k}b_k)t^n\big]\]
\end{lemma}
\begin{proof}
    A simple expansion of $fg$ and reordering \& factoring of terms willl yield the result.
\end{proof}
\begin{lemma}
    $B_k(x)=\sum_{i=0}^{k}{k\choose i}B_ix^{k-i}$
\end{lemma}
Now, we prove a series of lemmas that will help us derive neat properties for $\zeta$ and $B_m$.
\begin{proof}
    We have
    \[
        \frac{te^{xt}}{e^t-1}=\frac{t}{e^t-1}e^{xt}=\big(\sum_{m=0}^{\infty}
        B_m\cdot\frac{t^m}{m!}\big)\big(\sum_{m=0}^{\infty}\frac{(xt)^m}{m!}\big)
    \]
    \[
        =\sum_{n=0}^{\infty}\big(\sum_{m=0}^n \big(\frac{B_m}{m!}\cdot
        \frac{x^{m-n}}{(n-m)!}\big)\big)t^n    
        =\sum_{n=0}^{\infty}\big(\sum_{m=0}^nB_m\cdot x^{m-n}{n\choose m}\frac{1}{n!}\big)t^n
        \]\[=\sum_{n=0}^{\infty}\big(\sum_{m=0}^n{n\choose m}B_mx^{m-n}\big)\frac{t^n}{n!}
    \]
    Thus, 
    \[
        \sum_{k=0}^{\infty}B_k(x)\frac{t^k}{k!} = 
        \sum_{n=0}^{\infty}\big(\sum_{m=0}^n{n\choose m}B_mx^{m-n}\big)\frac{t^n}{n!}
    \]
    \[
        \implies B_k(x)=\sum_{i=0}^{k}{k\choose i}B_ix^{k-i}
    \]  
\end{proof}
\begin{lemma}
    For $k \geq 2, B_k(0) = B_k(1)$.
\end{lemma}
\begin{proof}
    When $x=0$, 
    \[
        \frac{t}{e^t-1}=\sum_{k=0}^{\infty}B_k(0)\frac{t^k}{k!}=B_0(0)+tB_1(0)+\sum_{k=2}^{\infty}B_k(0)\frac{t^k}{k!}
    \]
    \[
        1-\frac{1}{2}t+\sum_{k=2}^{\infty}B_k(0)\frac{t^k}{k!}  
    \]
    Thus, 
    \[
        \sum_{k=2}^{\infty}B_k(0)\frac{t^k}{k!} = \frac{t}{e^t-1}-1+\frac{1}{2}t=\frac{t+e^t(t-2)+2}{2(e^t-1)}
    \]
    When $x=1$,
    \[
        \frac{te^{t}}{e^t-1}=\sum_{k=0}^{\infty}B_k(1)\frac{t^k}{k!}=B_0(1)+tB_1(1)+\sum_{k=2}^{\infty}B_k(0)\frac{t^k}{k!}
    \]
    \[
        1+\frac{1}{2}t+\sum_{k=2}^{\infty}B_k(0)\frac{t^k}{k!}  
    \]
    Thus, 
    \[
        \sum_{k=2}^{\infty}B_k(0)\frac{t^k}{k!} = \frac{te^t}{e^t-1}-1-\frac{1}{2}t=\frac{t+e^t(t-2)+2}{2(e^t-1)}
    \]
    It follows that 
    \[
        \sum_{k=2}^{\infty}B_k(0)\frac{t^k}{k!}=\sum_{k=2}^{\infty}B_k(1)\frac{t^k}{k!} \implies B_k(0)=B_k(1)
    \]
\end{proof}
\begin{lemma}
    For $k \geq 1, B'_k(x)=kB_{k-1}(x)$, and $\int_0^1B_k(x)\,dx =0$
\end{lemma}
\begin{proof}
    We have 
    \[
        B_k(x)=\sum_{i=0}^{k}{k\choose i}x^{k-i}B_i\implies B'_k(x)= 
        \sum_{i=0}^{k-1}(k-i){k\choose i}x^{k-i-1}B_i
    \]
    \[
        =\sum_{i=0}^{k-1}(k-i)(\frac{k!}{(k-i)!i!})x^{k-i-1}B_i=
        k\sum_{i=0}^{k-1}\frac{(k-1)!}{(k-i-1)!i!}x^{k-i-1}B_i
    \]
    \[
        =k\sum_{i=0}^{k-1}{{k-1}\choose i}x^{k-i-1}B_i=kB_{k-1}(x)
    \]
    Now we compute the integral:
    \[
        \int_0^1B_k(x)\;dx=\int_0^1\sum_{i=0}^{k}{k\choose i}x^{k-i}B_i\; dx  
        ={\sum_{i=0}^k{k\choose i}B_i}\int_0^1x^{k-i}\; dx
        =\sum_{i=0}^k{k\choose i}B_i\frac{x^{k-i+1}}{k-i+1}\bigg |_0^1
    \]
    Continued algebraic manipulations yield 
    \[=
        \frac{1}{k-i+1}\sum_{i=0}^k{k\choose i}B_i  
        =\frac{1}{k-i+1}\sum_{i=0}^k\frac{k!}{(k-i)!i!}B_i 
    \]       
    \[=
        \frac{1}{k+1}\big(\sum_{i=0}^{k-1}{{k+1}\choose i}B_i+(k+1)B_k\big)
        =  \frac{1}{k+1}\big(\sum_{i=0}^{k-1}{{k+1}\choose i}B_i-\sum_{i=0}^{k-1}{{k+1}
        \choose i}B_i\big)=0
    \]
\end{proof}
We'll now consider $B_k(x)$ as a function on $[0, 1]$, and compute its
Fourier coefficients. We'll let $B_k(x)=\sum_{n\in\mathbb{Z}}\hat{f}(n)e^{2\pi inx}$. We also have the 
standard definition: $B_k(x)=\sum_{i=0}^k{k\choose i}x^{k-i}B_i$
Computing the fourier coefficients:
\[
    \hat{f}(n)=\int_0^1B_k(x)e^{-2\pi inx}\;dx = -\frac{1}{2\pi in}B_k(x)e^{-2\pi inx}\bigg|_0^1
    +\frac{1}{2\pi in}\int_0^1 B_k'(x)e^{-2\pi inx}\; dx 
\]
\[
    =-\frac{1}{2\pi in}\big(B_k(0)e^{-2\pi in}-B_k(1)\big)+\frac{1}{2\pi in}\int_0^1k
    B_{k-1}(x)e^{-2\pi inx}\;dx  
    = \frac{k}{2\pi in}\int_0^1B_{k-1}(x)e^{-2\pi inx}\; dx
\]
For the case of $k=1$, we know $B_k(x)=x-\frac{1}{2}$. This suggests the following 
formula: \[\hat{f}_k(n)=-(\frac{1}{2\pi in})^k\cdot(k!)\] This is easy to check by induction.\\\\
Now, we arrive at the grand result, a new expression for even zeta values:
\begin{theorem}
    For all $m\in\mathbb{Z}^+$, 
    \[
        \zeta(2m)=(-1)^m\pi^{2m}\frac{2^{2m-1}}{(2m-1)!}(-\frac{B_{2m}}{2m})
    \]
\end{theorem}
\begin{proof}
    We'll keep in mind the fourier results on $B_k(x)$. When $k$ is even, we see 
    $\hat{f}_k(n)=\hat{f}_k(-n)$.
    Thus, when $k$ is even, we get 
    \[
        B_k(x)=\sum_{n\in\mathbb{Z}}\hat{f}_k(n)e^{2\pi inx}=\sum_{n\in\mathbb{Z}^+}
        (\hat{f}_k(n)e^{2\pi inx}+\hat{f}_k(-n)e^{-2\pi inx}) = 
        \sum_{n\in\mathbb{Z}^+}
        (\hat{f}_k(n)e^{2\pi inx}+\hat{f}_k(-n)e^{-2\pi inx})
    \]
    \[
        \sum_{n\in\mathbb{Z}^+}\hat{f}_k(n)(e^{2\pi inx}+e^{-2\pi inx})
        = \sum_{n\in\mathbb{Z}^+}
        \hat{f}_k(n)(2\cos{(2\pi inx)})
    \]
    \[
        = \sum_{n\in\mathbb{Z}^+}-(\frac{1}{2\pi in})^k\cdot k!\cdot 2\cos(2\pi nx)=-(\frac{1}{2\pi i})^k\cdot
        k!\sum_{n\in\mathbb{Z}^+}\frac{1}{n^k}\cos{(2\pi nx)} 
    \]
    We also have $B_k(0)=(-\frac{1}{2\pi i})^k\cdot k!\sum_{n=1}^{\infty}\frac{1}{n^k}$.
    So 
    \[
        \sum_{k=1}^{\infty}\frac{1}{n^k}=B_k(0)\cdot\frac{(-2\pi i)^k}{k!}  
    \]
    Recall $k$ is even. So let $k=2m$ for positive $m$. Then we can reformulate:
    \[
        \zeta(2m)=B_{2m}(0)\cdot \frac{(-2\pi i)^{2m}}{(2m)!} = 
        (-1)^m\pi^{2m}\frac{2^{2m-1}}{(2m-1)!}(-\frac{B_{2m}}{2m})  
    \]
\end{proof}
