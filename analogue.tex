\section{Analogues to $L$-functions}
Now, we'll consider generalizations of the Zeta function, namely, $L$-functions. We define
Dirichlet Characters:
\begin{definition}
    Define a character \[\chi: (\zz/m\zz)^{\times}\rightarrow S^1 =\{z\in\mathbb{C} \big|\; |z|=1\}\]
where $\chi$ is a group homomorphism $(\chi(ab)=\chi(a)\chi(b)\;\; \forall a,b\in(\zz/m\zz)^{\times})$.
\end{definition}
We can extend $\chi$ to a function on $\mathbb{Z}$ by 
\[
    \chi:\zz \rightarrow \mathbb{C}  
\]
\[
    a \mapsto 
    \begin{cases} 
        \chi (a \;\text{mod}\; m) & (a,m) = 1 \\
        0 & (a,m)>1
    \end{cases}
\]
We'll note that we can see $\chi$ having period $m$. A primitive character mod $m$ means
that there is no proper divisor $d|m$ st there exists $\chi_d$ st $\chi=\chi_d\circ $(reduction $\pmod d$
on $(\zz/m\zz)^{\times}$)\\\\
The Dirichlet L-function $L(\chi,s)$ is defined by $L(\chi,s)=\sum_{n=1}^{\infty}\frac{\chi(n)}{n^s}$

Our goal is to generalize our results on the Zeta Function to Dirichlet $L$-functions.
\begin{remark}
A similar Euler-Product form can be derived here: For $\Re(s)>1$ and $\chi \pmod m$.
\[
    \prod_{p}(1-\chi(p)p^{-s})^{-1}  
\]
\end{remark}
\begin{definition}
    Define the Gauss sum 
    \[
        G(\chi)=\sum_{k=0}^{m-1}\chi(k)e^{2\pi ik/m}
    \]
\end{definition}
This sum is quite important for calculations going forward. We thus establish a crucial identity 
on the Gauss sum that will be useful later, namely that $|G(\chi,1)|^2=m$. We'll go through a line of lemmas
before reaching this identity.

\begin{lemma}
    Let $\chi$ be primitive $\pmod m$. Then for any divisor $m_1|m$, there exists 
    $c$ such that $(c,m)=1, c\equiv 1 \pmod {m_1}$ with $\chi(c)\neq 1$.
\end{lemma}
In fact, this can be thought of as an equivalent formulation of $\chi$ being
primitive. This helps us show:
\begin{lemma}
    If $\chi$ is primitive, then $G(\chi, a) = \chi(a)G(\chi, 1)$ for every $a \in \mathbb{Z}/m\mathbb{Z}$
\end{lemma}
\begin{proof}
    We separate into cases.\\\\
    \textbf{Case 1:} $(a,m)=1$\\\\
    We note that as $k$ runs through $\mathbb{Z}/{m\mathbb{Z}}$, $ak$ runs through
    $\mathbb{Z}/{m\mathbb{Z}}$. Let $y=ak$. Then $\chi(y)=\chi(a)\chi(k)\implies \chi(k)
    =\overline{\chi(a)}\chi(y)$
    So
    \[
        G(\chi,a)=\sum_{k\in\mathbb{Z}/m\mathbb{Z}}\chi(k)e^{2\pi iak/m} 
        =\sum_{k\in\mathbb{Z}/m\mathbb{Z}}\overline{\chi(a)}\chi(y)e^{2\pi iy/m} 
    \]
    \[
        =\overline{\chi(a)}G(\chi,1)  
    \]
    \textbf{Case 2:} $(a,m)>1$\\\\
    It is enough to show $(\chi, a)=0$. Suppose $m_1|m$. Let $d=(a,m)$, $a_1=\frac{a}{d}$, $m_1=\frac{m}{d}$. Choose 
    $c$ that satisfies the previous lemma. Then 
    \[
        \chi(c)G(\chi,a)=\sum_{k\in\mathbb{Z}/m\mathbb{Z}}\chi(ck)e^{2\pi iak/m}  
    \]
    Then these $y=ck$ hit all of $\mathbb{Z}/m\mathbb{Z}$. Recall $c\equiv 1\pmod {m_1}$. Then
    \[e^{2\pi iak/m}=e^{2\pi ika/m}=e^{2\pi ika_1/m_1}=e^{2\pi ikca_1/m_1}=e^{2\pi iya/m}\]
    So $\chi(c)G(\chi,a)=\sum_{y\in\mathbb{Z}/m\mathbb{Z}}\chi(y)e^{2\pi iya/m}=G(\chi,a)$.
    But $\chi(c)\neq 1$. So
    \[
        \chi(c)G(\chi,a)=G(\chi,a) \implies G(\chi,a)=0
    \]
    and we are done.
\end{proof}
Finally, we arrive at this well-known result:
\begin{theorem}
    Let $\chi$ be a primitive character$\pmod m$, Then $|G(\chi,1)|^2=m$
\end{theorem}
\begin{proof}
This can be done with simple computations:
\[
    |G(\chi,1)|^2=\overline{G(\chi,1)}G(\chi,1)=\sum_{k\in\mathbb{Z}/m\mathbb{Z}}\overline{\chi(k)}e^{-2\pi ia/m}G(\chi,1)
\]  
By the previous lemma,
\[
    =\sum_{k\in\mathbb{Z}/m\mathbb{Z}}e^{-2\pi i k/m}G(\chi,k)=\sum_{k\in\mathbb{Z}/m\mathbb{Z}}
    e^{-2\pi ik/m}\big(\sum_{j\in\mathbb{Z}/m\mathbb{Z}}\chi(j)e^{2\pi ikj/m}\big)
\]
\[
    =\sum_{k\in\mathbb{Z}/m\mathbb{Z}}\big(\sum_{j\in\mathbb{Z}/m\mathbb{Z}}
    e^{-2\pi ik/m}\chi(j)e^{2\pi ikj/m}\big)
    =\sum_{k\in\mathbb{Z}/m\mathbb{Z}}\sum_{j\in\mathbb{Z}/m\mathbb{Z}}
    \chi(j)e^{2\pi ik(j-1)/m}
\]
\[
    =\sum_{j\in\mathbb{Z}/m\mathbb{Z}}\chi(j)\sum_{k\in\mathbb{Z}/m\mathbb{Z}}e^{2\pi ik(j-1)/m}    
\]
Now, $\sum_{k\in\mathbb{Z}/m\mathbb{Z}}e^{2\pi ik(j-1)/m}$ is the sum over a geometric series.
When $j=1$, the sum goes to $m$; otherwise, the sum goes to $0$. Thus we have
\[
    |G(\chi,1)|^2=\sum_{j\in\mathbb{Z}/m\mathbb{Z}}\chi(j)\sum_{k\in\mathbb{Z}/m\mathbb{Z}}e^{2\pi ik(j-1)/m}
    =\chi(1)m=m
\]
\end{proof}

\begin{definition}
    We define an analogue for $\xi(s)$:
    \[\Lambda(\chi,s)=(\frac{m}{\pi})^{\frac{s}{2}}\Gamma(\frac{s+p}{2})L(\chi,s)\]
    where $p\in \{0,1\}$ and $\chi(-1)=(-1)^p$.
\end{definition}

We want to extend some of the complex properties to our analogue. Let's
build up some theory before we arrive at a holomorphicity theorem for $\Lambda$

We have 
    \[
        \Gamma(\frac{s+p}{2})=\int_0^{\infty}e^{-y}y^{\frac{s+p}{2}}\frac{dy}{y}
        = \int_0^{\infty}e^{-\pi n^2t/m}(\frac{\pi n^2t}{m})^{\frac{s+p}{2}}\frac{dt}{t}
    \]
    Thus, 
    \[
        (\frac{m}{\pi})^{\frac{s+p}{2}}\Gamma(\frac{s+p}{2}\chi(n)n^{-s})
        =\chi(n)\int_0^{\infty}e^{-\pi n^2 t/m}n^pt^{\frac{s+p}{2}}\frac{dt}{t}
    \]  
    As before, we can take the sum for $n=1,\dots,\infty$ to obtain:
    \[
        (\frac{m}{\pi})^{p/2}\cdot\Lambda(\chi,s)=\sum_{n=1}^{\infty}\chi(n)\int_0^{\infty}e^{-\pi n^2 t/m}n^pt^{\frac{s+p}{2}}\frac{dt}{t}
    \]
    \[
        =\int_0^{\infty}(\sum_{n=1}^{\infty}\chi(n)n^pe^{-\pi n^2t/m})t^{\frac{s+p}{2}}\frac{dt}{t}  
    \]
    Now we define some generalization of the $\Theta$ function, which will be crucial as we move forward
    \begin{definition}
        Set 
        \[
            \Theta(\chi,y)=\sum_{n\in\mathbb{Z}}\chi(n)n^pe^{-\pi n^2y/m}  
        \]
    \end{definition}
    Thus, 
    \[
        (\frac{m}{\pi})^{p/2}\cdot\Lambda(\chi,s)=\int_0^{\infty}\frac{\Theta(\chi,t)}{2}t^{\frac{s+p}{2}}\frac{dt}{t}   
    \]
    \begin{theorem}
        Again, we have a neat identity for $\Theta$:
        \[
            \Theta(\chi,1/y)=\frac{G(\chi)}{i^p\sqrt{m}}y^{p+\frac{1}{2}}\Theta(\overline{\chi},y)
        \]
    \end{theorem}
    We now present a holomorphicity theorem for $\Lambda$
\begin{theorem}
    Let $\chi$ be a non-trivial primitive Dirichlet Character. Then 
    $\Lambda(\chi,s)$ extends to a holomorphic function on all of $\mathbb{C}$
    This extension satisfies the functional equation 
    \[
        \Lambda(\chi,s)=\frac{G(\chi)}{i^p\sqrt{m}}\Lambda(\overline{\chi},1-s)  
    \]
\end{theorem}
\begin{proof}
    We have
    \[
        (\frac{m}{\pi})^{p/2}\cdot\Lambda(\chi,s)=\int_0^1+\int_1^{\infty}=\int_1^{\infty}\frac{\Theta(\chi,1/y)}{2}y^{-(\frac{s+p}{2})}\frac{dy}{y}+\int_1^{\infty}\frac{\Theta(\chi,t)}{2}t^{\frac{s+p}{2}}\frac{dt}{t}
    \]
    \[
        =\frac{G(\chi)}{i^p\sqrt{m}}\int_1^{\infty}\frac{\Theta(\overline{\chi},y)}{2}y^{p+\frac{1}{2}-(\frac{s+p}{2})}\frac{dy}{y}+\int_1^{\infty}\frac{\Theta(\chi,t)}{2}t^{\frac{s+p}{2}}\frac{dt}{t}    
    \]
    which is holomorphic on all of $\mathbb{C}$. Multiplying by $\frac{i^p\sqrt{m}}{G(\chi)}$, we get 
    \[
        (\frac{m}{\pi})^{p/2}\frac{i^p\sqrt{m}}{G(\chi)}\cdot\Lambda(\chi,s)=\int_1^{\infty}\frac{\Theta(\overline{\chi},y)}{2}y^{\frac{1-s+p}{2}}\frac{dy}{y}+\frac{i^p\sqrt{m}}{G(\chi)}\int_1^{\infty}\frac{\Theta(\chi,t)}{2}t^{\frac{s+p}{2}}\frac{dt}{t}    
    \]
    On the other hand, we can take 
    \[
        (\frac{m}{\pi})^{p/2}\Lambda(\overline{\chi},1-s)=\int_0^{\infty}
        \frac{\Theta(\overline{\chi},t)}{2}t^{\frac{1-s+p}{2}}\frac{dt}{t}
        =\int_1^{\infty} \frac{\Theta(\overline{\chi},t)}{2}t^{\frac{1-s+p}{2}}\frac{dt}{t}
        +\int_1^{\infty} \frac{\Theta(\overline{\chi},\frac{1}{y})}{2}y^{-\frac{1-s+p}{2}}\frac{dy}{y}
    \]
    \[
        = \int_1^{\infty} \frac{\Theta(\overline{\chi},t)}{2}t^{\frac{1-s+p}{2}}\frac{dt}{t} 
        +\frac{G(\overline{\chi})}{i^p\sqrt{m}}\int_1^{\infty}\frac{\Theta(\chi,y)}{2}y^{p+\frac{1}{2}-(\frac{1-s+p}{2})}\frac{dy}{y}
    \]
    We notice that the $(\frac{m}{\pi})^{p/2}$ cancel; equating terms, we get
    $\frac{i^p\sqrt{n}}{G(\chi)}\Lambda(\chi,s)$ agrees with $\Lambda(\overline{\chi},1-s)$ as long as 
    $\frac{i^p\sqrt{m}}{G(\chi)}=\frac{G(\overline{\chi})}{i^p\sqrt{m}}$. If we can show this identity, 
    we are done.\\\\
    Recall $|G(\chi)|=\sqrt{m} \implies G(\chi)\overline{G(\chi)}=m$. Using this, note
    \[
        \overline{G(\chi)}=\sum_{k=0}^{m-1}\overline{\chi}(k)e^{-2\pi ik/m}=\chi(-1)\sum_{k=0}^{m-1}
        \overline{\chi}(k)e^{2\pi ik/m}
    \]
    \[
        =\chi(-1)G(\overline{\chi})=(-1)^pG(\overline{\chi})  
    \]
    Thus 
    \[
        \frac{i^p\sqrt{m}}{G(\chi)}=\frac{i^p\sqrt{m}\cdot\overline{G(\chi)}}{G(\chi)\cdot\overline{G(\chi)}}    
        =\frac{G(\overline{\chi})}{i^p\sqrt{m}}
    \]
    and we're done.
\end{proof}





